% !Mode:: "TeX:UTF-8"

%%% 此部分需要自行填写: (1) 中文摘要及关键词 (2) 英文摘要及关键词
%%%%%%%%%%%%%%%%%%%%%%%%%%%%%
%%% -------------  英文封面 (无需改动)-------------   %%%
%%%%%%%%%%%%%%%%%%%%%%%%%%%%%
\thispagestyle{empty}
\renewcommand{\baselinestretch}{1.5}  %下文的行距
\vspace*{0.5cm}
\begin{center}
{\Large \bf BACHELOR'S DEGREE THESIS \\[1ex] OF WUHAN UNIVERSITY }
\end{center}
\vspace{2.5cm}
\begin{center}{\zihao{2} \the\Etitle \par}\end{center}

\vfill

\begin{center}
\zihao{4}
\begin{tabular}{ r l }
 School (Department): & {\sc \the\Eschoolname}\\
  Major:          &   {\sc\the\Emajor}  \\
 Candidate:      &  {\sc \the\Eauthor}      \\
 Supervisor:     &  {\sc \the\Esupervisor}
\end{tabular}

\vspace*{2cm}
\begin{center}
   \ifprint % 文档打印, 使用黑白校徽.
  \includegraphics[height=4cm]{whu.eps}       %%  黑白的.
  \else
  \includegraphics[height=4cm]{whulogo.eps} %%  彩色的.
  \fi
\end{center}


\zihao{-2}
%\the\Schoolname\\
{\sc Wuhan University}

\vspace*{1.0cm}

\the\Edate

\end{center}
%%% 郑重声明部分无需改动

%%%---- 郑重声明 (无需改动)------------------------------------%
\newpage
\vspace*{20pt}
\begin{center}{\ziju{0.8}\textbf{\songti\zihao{2} 郑重声明}}\end{center}
\par\vspace*{30pt}
\renewcommand{\baselinestretch}{2}

{\zihao{4}%

本人呈交的学位论文, 是在导师的指导下, 独立进行研究工作所取得的成果,
所有数据、图片资料真实可靠. 尽我所知, 除文中已经注明引用的内容外,
本学位论文的研究成果不包含他人享有著作权的内容.
对本论文所涉及的研究工作做出贡献的其他个人和集体,
均已在文中以明确的方式标明. 本学位论文的知识产权归属于培养单位.\\[2cm]

\hspace*{1cm}本人签名: $\underline{\hspace{3.5cm}}$
\hspace{2cm}日期: $\underline{\hspace{3.5cm}}$\hfill\par}
%------------------------------------------------------------------------------
\baselineskip=23pt  % 正文行距为 23 磅
%------------------------------------------------------------------------------





%%======中文摘要===========================%
\begin{cnabstract}
本文主要介绍和讨论了武汉大学本科毕业论文的~\LaTeX~模板.
指明了编译方法, 强调了公式排版的一些细节问题, 也指出了一些常见的排版错误.



\end{cnabstract}
\par
\vspace*{2em}


%%%%--  关键词 -----------------------------------------%%%%%%%%
%%%%-- 注意: 每个关键词之间用“;”分开,最后一个关键词不打标点符号
\cnkeywords{毕业论文; \LaTeX{}; 模板;  }


%%====英文摘要==========================%


\begin{enabstract}
This thesis is a study on the theory of \dots.

\end{enabstract}
\par
\vspace*{2em}

%%%%%-- Key words --------------------------------------%%%%%%%
%%%%-- 注意: 每个关键词之间用“;”分开,最后一个关键词不打标点符号
 \enkeywords{\LaTeX{};  }
